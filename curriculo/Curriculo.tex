%%%%%%%%%%%%%%%%%%%%%%%%%%%%%%%%%%%%%%%%%
% Journal Article
% LaTeX Template
% Version 1.4 (15/5/16)
%
% This template has been downloaded from:
% http://www.LaTeXTemplates.com
%
% Original author:
% Frits Wenneker (http://www.howtotex.com) with extensive modifications by
% Vel (vel@LaTeXTemplates.com)
%
% License:
% CC BY-NC-SA 3.0 (http://creativecommons.org/licenses/by-nc-sa/3.0/)
%
%%%%%%%%%%%%%%%%%%%%%%%%%%%%%%%%%%%%%%%%%

%----------------------------------------------------------------------------------------
%	PACKAGES AND OTHER DOCUMENT CONFIGURATIONS
%----------------------------------------------------------------------------------------

\documentclass[a4paper]{article}
\usepackage{framed}
\usepackage{mdframed}

\usepackage{blindtext} % Package to generate dummy text throughout this template 

\usepackage[sc]{mathpazo} % Use the Palatino font
\usepackage[T1]{fontenc} % Use 8-bit encoding that has 256 glyphs
\usepackage{parskip} % espaçamento de paragrafos
\setlength{\parindent}{0pt}
\linespread{1.3} % Line spacing - Palatino needs more space between lines


\usepackage[english]{babel} % Language hyphenation and typographical rules

\usepackage[hmarginratio=1:1,top=10mm, bottom=8mm,right=13mm, left=13mm,columnsep=10pt]{geometry} % Document margins
\usepackage[hang, small,labelfont=bf,up,textfont=it,up]{caption} % Custom captions under/above floats in tables or figures
\usepackage{booktabs} % Horizontal rules in tables

\usepackage{lettrine} % The lettrine is the first enlarged letter at the beginning of the text

\usepackage{enumitem} % Customized lists
\setlist[itemize]{noitemsep} % Make itemize lists more compact

\usepackage{abstract} % Allows abstract customization
\renewcommand{\abstractnamefont}{\normalfont\bfseries} % Set the "Abstract" text to bold
\renewcommand{\abstracttextfont}{\normalfont\small\itshape} % Set the abstract itself to small italic text


\usepackage{titlesec} % Allows customization of titles
\titleformat{\section}[block]{\large}{\thesection}{0.5em}{\textbf{/ / }\textbf}{} % Change the look of the section titles
\titlespacing{\section}{0pt}{-3pt}{5pt}
\titleformat{\subsection}[block]{\large}{\thesubsection.}{1em}{\textbf}{} % Change the look of the section titles

\usepackage{titling} % Customizing the title section

\usepackage{hyperref} % For hyperlinks in the PDF
\hypersetup{
  colorlinks = true,
   allcolors = blue,
  allbordercolors=0 1 0
}

\usepackage{fontspec}
\setmainfont{Montserrat}
    
\usepackage{fontawesome5}
\usepackage{multicol}


%----------------------------------------------------------------------------------------
%	TITLE SECTION
%----------------------------------------------------------------------------------------






%----------------------------------------------------------------------------------------
\setcounter{secnumdepth}{0}

\begin{document}
\begin{multicols}{2}



  % Print the title
  {\fontsize{40pt}{46pt}\selectfont{\textbf{GABRIEL\\VICTOR}}}\\
  \\
  \normalsize{{\Large{\faIcon{city}}} Belo Horizonte, Minas Gerais - Brazil}
  %----------------------------------------------------------------------------------------
  %	ARTICLE CONTENTS
  %----------------------------------------------------------------------------------------

  \begin{framed}
    \section{CONTATO}
    \begin{itemize}[itemsep=1ex]
      \item {\Large{\faIcon{envelope}}} \href{mailto:gabriel.victorc13@gmail.com}{gabriel.victorc13@gmail.com}
      \item {\Large{\faIcon{linkedin}}} \href{https://linkedin.com/in/gabriel-victorc/}{linkedin.com/in/gabriel-victorc/}
      \item {\Large\faIcon{phone-square-alt}} (31) 99556-4084
      \item {\Large\faIcon{github}} \href{https://github.com/G4BR-13-L}{github.com/G4BR-13-L}
    \end{itemize}
  \end{framed}

  \begin{framed}
    \section{PERFIL}

    Destemido, inconformado e com vontade de mudar o mundo. E se o preço a se pagar para fazer a diferença é aprender uma linguagem, framework ou tecnologia nova, que assim seja.

    Acredito que os melhores trabalhos são a soma de atividades executadas de maneira inteligente e estratégica que economizam tempo e recursos e que sempre são uma disrupção. Ou seja: Inovação. Que na prática não são soluções que mudam o mundo, e sim pequenas coisas que fazem a diferença no dia a dia do time.
  \end{framed}

  \begin{framed}
    \section{HABILIDADES}
    \begin{tabular}{r|p{5cm}}
    \end{tabular}
    HTML, CSS, JavaScript, React, React Native, Expo, TypeScript, Sass,  NextJs, Java, Firebase, Arch Linux, inglês avançado, WordPress, Typescript, Scrum, O.K.Rs, Git, Docker, API Rest
  \end{framed}

  \begin{framed}
    \section{CONHECIMENTOS}
    \begin{itemize}
      \item Diagramação de UML;
      \item Desenvolvimento Dirigido por Testes(TDD)
      \item Princípios de qualidade de software e Modularidade
      \item Metodologias Ágeis;
    \end{itemize}
  \end{framed}

  \begin{framed}
    \section{FORMAÇÃO}

    \normalsize\textbf{BACHAREL EM ENGENHARIA DE SOFTWARE}\\
    \footnotesize{PUC MINAS | 2021 - 2024}

    \normalsize\textbf{TÉCNICO EM TELECOMINICAÇÕES E REDES DE COMPUTADORES}\\
    \footnotesize{E. E. TÉCNICO INDUSTRIAL PROFESSOR FONTES | 2019 - 2020}
  \end{framed}

  \begin{framed}
    \section{PORTIFOLIO}
      \small\subsection{Sistema de Pedidos de Hamburgueria}
      \footnotesize{NextJs, TypeScript, Sass, Firebase}\\
      \href{https://artgo-81447.web.app/}{artgo-81447.web.app/}

      \small\subsection{Listfy - Trabalho Interdisciplinar}
      \footnotesize{HTML, CSS, JavaScript}\\
      \href{https://icei-puc-minas-pples-ti.github.io/plf-es-2021-1-ti1-7924100-dificuldades-do-ead/}{icei-puc-minas-dificuldades-do-ead/}

      \small\subsection{Portifolio completo}
      \footnotesize{NextJs, Firebase, TypeScript, Sass, Markdown}\\
      \href{https://gabrielvictor.web.app}{https://gabrielvictor.web.app}
  \end{framed}









  %\twocolumn[
  %\begin{@twocolumnfalse}
  \begin{framed}
    \section{EXPERIÊNCIA PROFISSIONAL}
    \begin{itemize}
      \item Trabalho Remoto;
      \item Trabalho Voluntário;

    \end{itemize}

    \subsection{ \faIcon{rocket} IPRO Jr. Soluções em Engenharia e Arquitetura}

    \begin{tabular}{r|p{5cm}}
      \emph{Presente}   & \textsc{Coordenador e Gerente de projetos em Sistemas de Informação}                                                                                                                                                                                                                                          \\
      \textsc{Mar 2022} & \footnotesize{Levantamento, análise e especificação de requisitos; metrificação, acompanhamento e execução de projetos como Scrum Master; manutenção de KanBans com Pipefy, Notion e planilhas, negociação e colaboração com o cliente.}                                                                      \\
      \multicolumn{2}{c}{}                                                                                                                                                                                                                                                                                                              \\

      \emph{Mar 2022}   & \textsc{Dev React, React Native e Wordpress }                                                                                                                                                                                                                                                                 \\
      \textsc{Set 2021} & \footnotesize{Pesquisas de viabilidade, elaboração de propostas de projeto, negociação de projetos, levantamento de requisitos, desenvolvimento de sites em WordPress, desenvolvimento de ferramentas internas em JavaScript, React, React Native e Firebase. Elaboração de projetos de interface com Figma.} \\
      \multicolumn{2}{c}{}                                                                                                                                                                                                                                                                                                              \\

      \emph{Set 2022}   & \textsc{Trainee}                                                                                                                                                                                                                                                                                              \\
      \textsc{Ago 2021} & \footnotesize{Resolução de cases,desenvolvimento de pitch, treinamento em pipefy, palestras com representantes do Núcleo Central}                                                                                                                                                                             \\
      \multicolumn{2}{c}{}                                                                                                                                                                                                                                                                                                              \\
    \end{tabular}
  \end{framed}
  %\end{@twocolumnfalse}

\end{multicols}
\end{document}
